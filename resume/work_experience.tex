%%%%%% TITLE %%%%%%

\EN{\cvsection[4]{Work experience}}
\IT{\cvsection{Esperienze lavorative}}


%%%%%% ENTRIES %%%%%%

\newcommand{\workPhD}{
  \cventry
  {
    \EN{Network intrusion detection on IoT systems, building automation, and critical infrastructures}
    \IT{Detezione di intrusioni in reti Internet of Things (IoT), building automation, ed infrastrutture critiche}
  }
  {
    \EN{PhD Candidate in Cyber Security -- Dept. of Computer Science, TU/e}
    \IT{Dottorando in Cyber Security -- Dip. di Computer Science, TU/e}
  }
  {Eindhoven (NL)} % Location
  {2015--2020} % Date(s)
  {
    \begin{cvitems}
      \item {
        \EN{Published research articles jointly with {\bf Forescout}, integrating in their Agile team and addressing real-world industry issues on IoT/OT network visibility, data-driven monitoring, and actionable intrusion detection}
        \IT{Pubblicato articoli scientifici in collaborazione con {\bf Forescout}, integrandomi nel loro team Agile, e affrontando problematiche reali di visibilità di reti IoT/OT, monitoraggio data-driven, e usabilità di IDS}
      }
      \item {
        \EN{Developed network monitoring sensors for embedded Linux devices as part of {\bf EU H2020 CITADEL} project}
        \IT{Sviluppato sensori di monitoraggio di rete per embedded Linux come parte del progetto europeo {\bf H2020 CITADEL}}
      }
      \item {
        \EN{Assessed a SCADA network for {\bf ASML} and communicated results to management; responsibly disclosed vulnerabilities to ICS-CERT}
        \IT{Valutato la sicurezza della rete SCADA per {\bf ASML}; comunicato i risultati al management; divulgato responsabilmente le vulnerabilità ad ICS-CERT}
      }
      \item {
        \EN{Managed \char`~100 VMs as lab administrator and assistant instructor of a penetration testing fundamentals course}
        \IT{Amministrato un laboratorio di circa 100 VM in qualità di aiuto istruttore per un corso di basi di penetration testing}
      }
    \end{cvitems}
  }
}


\newcommand{\workDII}{
  \cventry%
  {
    \EN{Modelization and control of power electronics for Vanadium Redox Flow Batteries}
    \IT{Modellizzazione e controllo dell'elettronica di potenza per Batterie a Flusso Redox di Vanadio}
  }
  {
    \EN{Researcher on Electrochemical Energy Storage -- Dept. of Industrial Engineering, UniPD}
    \IT{Ricercatore in Accumulo di Energia Elettrochimica -- Dip. di Ing. Industriale, UniPD}
  }
  {Padova (IT)}
  {2014--2015}
  {
    \begin{cvitems}
      \item {
        \EN{Coordinated a small research team to design and model a modular 3kW VRFB prototype and its power supply}
        \IT{Coordinato un piccolo team di ricerca per progettare e modellizzare un impianto di accumulo VRFB da 3kW con convertitore di potenza}
      }
      \item {
        \EN{Fully automated the layout configuration choice, multiphysical modelization, and LaTeX report generation for any prototype operating point}
        \IT{Completamente automatizzato la scelta dell'architettura, la modellizzazione multi-fisica e la generazione di report LaTeX illustrati}
      }
    \end{cvitems}
  }
}


\newcommand{\workPowerControlSystems}{
  \cventry%
  {
    \EN{Technological transfer on power conversion for batteries}
    \IT{Trasferimento tecnologico su conversione di potenza per impianti di accumulo}
  }
  {
    \EN{Internship on Power Electronics -- Power Control Systems S.r.l.}
    \IT{Tirocinio in Elettronica di Potenza -- Power Control Systems S.r.l.}
  }
  {San Vendemiano (IT)}
  {2015}
  {
    \begin{cvitems}
      \item {
        \EN{Prototyped a portable potentiostat for electrocatalysis; performed functional and safety tests on DC/DC converters}
        \IT{Prototipato un potenziostato portatile per elettrocatalisi; svolto test funzionali e di sicurezza su convertitori DC/DC per impianti di accumulo}
      }
    \end{cvitems}
  }
}


\newcommand{\workDISC}{
  \cventry%
  {
    \EN{Construction and characterization of a Vanadium Redox Flow Battery prototype cell}
    \IT{Realizzazione e caratterizzazione di un prototipo di Vanadium Redox Flow Battery}
  }
  {
    \EN{Internship on Electrochemistry -- Dept. of Chemical Sciences, UniPD}
    \IT{Tirocinio in Elettrochimica -- Dip. di Scienze Chimiche, UniPD}
  }
  {Padua (IT)}
  {2013--2014}
  {
    \begin{cvitems}
      \item {
        \EN{Replicated the state of the art for a VRFB prototype cell including synthesis of components, assembly, and tests}
        \IT{Riprodotto lo stato dell'arte per una cella VRFB, inclusa sintesi di componenti, assemblaggio e collaudo}
      }
    \end{cvitems}
  }
}


\newcommand{\workTeletronic}{
  \cventry%
  {
    \EN{Development of PLC logic and HMI panel for a co-generation biomass power plant}
    \IT{Sviluppo di logica PLC e pannello HMI per una centrale di co-generazione a biomassa}
  }
  {
    \EN{Internship on Industrial Automation -- Teletronic S.r.l.}
    \IT{Tirocinio in Automazione Industriale -- Teletronic S.r.l.}
  }
  {Vigonza (IT)}
  {2008}
  {
    \begin{cvitems}
      \item {
        \EN{Implemented PID controllers on Siemens PLCs, satisfying requirement specifications and hardware constraints}
        \IT{Implementato controllori PID per PLC Siemens, rispettando specifiche dei requisiti e limiti hardware}
      }
      \item {
        \EN{Trained other personnel on PLC programming}
        \IT{Istruito altri colleghi nella programmazione di PLC}
      }
    \end{cvitems}
  }
}


% \cventry%
%     {Università degli Studi di Padova}
% 	{State exam for engineering profession}
% 	{Padua, IT}
% 	{2015}
% 	{}


%%%%%% LAYOUT %%%%%%

\begin{cventries}

  \workPhD

  \workDII

  \workPowerControlSystems

  \workDISC

  \workTeletronic

\end{cventries}
