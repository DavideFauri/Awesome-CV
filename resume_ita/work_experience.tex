\cvsection{Esperienze lavorative}

\begin{cventries}
    \cventry%
        {Intrusion Detection in reti Internet of Things (IoT), Building Automation, ed infrastrutture critiche.} % Institute
        {Dottorato in Cyber Security -- Dept. of Computer Science, TU/e} % Title
        {Eindhoven (NL)} % Location
        {2015--2020} % Date(s)
        {\begin{cvitems}
            \item Collaborato con Forescout/SecurityMatters per diverse pubblicazioni su problematiche reali di gestione di rete e intrusion detection.
            \item Svolta una valutazione della sicurezza della rete SCADA di ASML, e divulgato responsabilmente le vulnerabilità scoperte presso ICS-CERT.
            \item Sviluppato sensori di monitoraggio di rete per il progetto europeo CITADEL (Horizon 2020).
            \item Amministratore di laboratorio ed istruttore per il corso \emph{Hacker's Hut} (basi di penetration testing).
            \item Data Science pratica: detezione di anomalie, reti di Bayes, regole di associazione, Explainable Machine Learning, modelli lineari generalizzati (GLM).
        \end{cvitems}}

	% \cventry%
    %     {Università degli Studi di Padova}
	% 	{State exam for engineering profession}
	% 	{Padua, IT}
	% 	{2015}
	% 	{}

    \cventry%
        {Modellizzazione e controllo dell'elettronica di potenza per Batterie a Flusso Redox di Vanadio.}
        {Ricercatore in accumulo di energia elettrochimica -- Dip. Ingegneria Industriale, UniPD}
        {Padova (IT)}
        {2014--2015}
        {\begin{cvitems}
            \item Coordinato un piccolo team di ricerca per progettare e modellizzare un impianto modulare di accumulo VRFB da 3kW, e il suo convertitore di potenza.
            \item Automatizzato la scelta dell'architettura e la modellizzazione multi-fisica a partire da un qualsiasi punto di lavoro del prototipo.
            \item Automatizzato la generazione di report LaTeX illustrati per un qualsiasi punto di lavoro.
        \end{cvitems}}

    \cventry%
        {Trasferimento tecnologico su conversione di potenza per impianti di accumulo}
        {Tirocinio in elettronica di potenza -- Power Control Systems S.r.l.}
        {San Vendemiano (IT)}
        {2015}
        {\begin{cvitems}
            \item Svolto test funzionali e di sicurezza su convertitori DC/DC per impianti di accumulo.
            \item Costruito un potenziostato portatile per elettrocatalisi e voltammetria ciclica.
        \end{cvitems}}

    \cventry%
        {Realizzazione e caratterizzazione di un prototipo di cella Vanadium Redox Flow Battery}
        {Tirocinio in elettrochimica -- Dip. di Scienze Chimiche, UniPD}
        {Padova (IT)}
        {2013--2014}
        {\begin{cvitems}
            \item Sintetizzato e preparato elettroliti liquidi, elettrodi, membrane a scambio cationico.
            \item Assemblato, connesso e testato un prototipo allo stato dell'arte di cella VRFB.
        \end{cvitems}}

    \cventry%
        {Sviluppo di logica PLC e pannello HMI per una centrale di co-generazione a biomassa}
        {Tirocinio in automazione industriale -- Teletronic S.r.l.}
        {Vigonza (IT)}
        {2008}
        {\begin{cvitems}
            \item Ideato e realizzato una soluzione ``key-value'' per aggiornare le soglie di allarme.
            \item Implementato controllori PID per PLC Siemens Step7.
            \item Istruito altro personale dipendente nella programmazione di PLC.
        \end{cvitems}}

\end{cventries}
